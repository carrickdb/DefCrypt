%!TEX root = main.tex

\section{Introduction}\label{sec-intro}

The project aims to create a repository for cryptographic definitions. The intent is to write definitions in a unified, precise way. 

Some of the definitions are novel. We plan to incorporate them into research papers over time. 

Latex files can be downloaded and used. For definitions that are novel, please check with us first. Use your judgement with regard to citation.

\heading{Syntax.} We use what we call the dot syntax. For example a function family $\FF$ specifies its keyspace $\Keys{\FF}$, domain $\Dom{\FF}$ and range $\Rng{\FF}$ as shown, so that $\FF \Colon \Keys{\FF} \cross \Dom{\FF}\to \Rng{\FF}$. This means that, once we write $\FF$, we can invoke the associated sets directly via the dot extensions. 

Traditionally, the syntax of a primitive is a tuple, like a PKE scheme is the triple of key generation, encryption and decryption algorithms. Now, the scheme is a single object with the algorithms specified after the dots via standard extensions. 

It makes notation compact. It also allows expansion. If you want to refer to some unconventional element (for example, length of the randomness), add a dot extension. It also moves us closer to a programming language rendition, which is valuable both pedagogically (PlayCrypt) and pragmatically.

\heading{Security.} A security definition will associate to a primitive, notion and adversary (1) a game and (2) an advantage function. Security means the advantage of any poly-time adversary is negligible. This provides a uniform template for definitions. It allows both concrete and asymptotic statements. 

The above referred to non-simulation style definitions, but these also will be handled via games and advantages that depend also on a simulator. What security means is that for every poly-time adversary, there is a simulator making the advantage negligible.


\heading{Motivation.} Definitions allow us to formalize and capture security goals. They are the basis for reduction-based (also called provable security) cryptography. They are also valuable in cryptanalysis as targets for attack. 

The proliferation of definitions, written in different style, with different degrees of precision, can make the academic literature hard to navigate. Our intent is a more structured approach.

Our definitions are proof-friendly. They make it easier to write reduction-based proofs by naming all oracles in games.

When working on a problem or writing a paper, one can feel confusion, or that it is not clear how to write something precisely and in detail. The fault often lies with the definitions. We think ours will help.



 




