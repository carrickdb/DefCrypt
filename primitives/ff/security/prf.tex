%!TEX root = ../ff.tex

\heading{Pseudorandom Function}
We present GGM's definition of pseudorandom function security~\cite{GolGolMic86}. The formalization is based on game $\gPRF{\FF}(\advA)$ of \figref{fig-ff-prf}, associated to family of functions $\FF$ and adversary $\advA$. The game initially samples a random challenge bit $b$, with $b=1$ indicating it is in ``real'' mode and $b=0$ that it is in ``random" mode. As per our conventions noted in \secref{sec-defs},  the sets $V,W$ are assumed initialized to the empty set, and the integer $v$ is assumed initialized to $0$. Now the adversary $\advA$ also has access to an evaluation oracle $\RoRO$ that takes an input $x$. The oracle either returns a uniformly random element from $\Rng{\FF}$ (for $b=0$), or an evaluation under $\Ev{\FF}$ using the key (for $b=1$). The oracle checks that $\advA$ does not an input $x$. As a last step, the adversary outputs a bit $b'$ that can be viewed as a guess for $b$.  The advantage of adversary $\advA$ in breaking the mu-ind security of   $\se$ is defined as $\prfAdv{\FF}{\advA} = 2\Pr[\gPRF{\FF}(\advA)]- 1$.

\begin{figure} [t]
\oneCol{0.23}
{
\gameName{$\gPRF{\FF}(\advA)$} 
$b \getsr \bits$ ;
$b' \getsr \advA^{\RoRO}$\\
Return $(b' = b)$ \medskip

\underline{$\FnO(x)$}\\[2pt]
If ($x \in \setX$) then return $\bot$ \\
$y_1\gets \Ev{\FF}(\fkey,x)$\\
$y_0\getsr \Rng{\FF}$ \\
$\setX \gets \setX\cup\{x\}$ \\
Return $y_b$
}
\vspace{-2ex}
\caption{Game defining prf security of function family $\FF$.}
\label{fig-ff-prf}
\hrulefill
\end{figure}

