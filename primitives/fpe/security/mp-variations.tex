\heading{Message Privacy Variations.}
As mentioned above, different classes of $\sampW$ give rise to different types of attack.

To specify an adaptive chosen-plaintext attack, one may, for instance, define $\choice{\sampW}$ to merely output the tweak-message pair with which $\advA$ has queried the oracle as shown in \figref{fig-fpe-ch}. (Here $\epsilon$ is the empty string, because there is nothing for $\sampW$ to leak to $\advA$.) On the other hand, to define a static adversary, $\choice{\sampW}$ could produce its output independently of its input $\mu$. By varying what $\choice{\sampW}$ leaks via $\ell$ we can capture known-plaintext attacks, unknown-plaintext attacks, and variations thereof.

\begin{figure} [t]
\begin{center}
\fbox
{
\begin{pchstack}
\procedure{$\choice{\sampW}(\mu, \sigma)$}
  {
    (T, \msg) \gets \mu \\
    \pcreturn (T, \msg, \sigma, \epsilon)
  }
    \pchspace
\end{pchstack}
}
\end{center}
\vspace{-2ex}
\caption{ Example of $\choice{\sampW}$ that specifies an adaptive CPA attack in the message privacy framework. }
\label{fig-fpe-ch}
\hrulefill
\end{figure}
