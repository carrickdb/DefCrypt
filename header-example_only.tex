\usepackage[letterpaper,hmargin=1in,vmargin=1in]{geometry}
       \usepackage{multirow}
\usepackage{setspace}
\usepackage{xspace}
\usepackage{wrapfig}
\usepackage{url}
\usepackage{latexsym}
\usepackage{graphicx}
\usepackage{tikz}
\usetikzlibrary{shapes, arrows, shadows}
\usetikzlibrary{trees}
\usetikzlibrary{arrows}
\usetikzlibrary{shapes.misc}
\usepackage{amscd,amsmath}
\usepackage{amsfonts}
\usepackage{color}
\usepackage{float}
\usepackage{longtable}
% \usepackage{amsthm}
% conflict with proof environment
\usepackage{color,colortbl}

\DeclareMathAlphabet{\mathsl}{OT1}{cmr}{m}{sl}
\DeclareMathAlphabet{\mathsc}{OT1}{cmr}{m}{sc}
\DeclareMathAlphabet{\mathslbf}{OT1}{cmr}{bx}{sl}
\DeclareFontFamily{OT1}{pzc}{}
\DeclareFontShape{OT1}{pzc}{m}{it}%
             {<-> s * [1] pzcmi7t}{}
\DeclareMathAlphabet{\mathscript}{OT1}{pzc}{m}{it}
\setlength{\fboxsep}{1pt}

% ====================================================================

\renewcommand{\topfraction}{0.99} 
\renewcommand{\bottomfraction}{0.8}	
\setcounter{topnumber}{2}
\setcounter{bottomnumber}{2}
\setcounter{totalnumber}{2}
\setcounter{dbltopnumber}{2}    
\renewcommand{\dbltopfraction}{0.9}	
\renewcommand{\textfraction}{0.07}	
\renewcommand{\floatpagefraction}{0.7}
\renewcommand{\dblfloatpagefraction}{0.7}	
\hyphenation{whe-ther}

\newtheorem{thm}{Theorem}[section]
\newtheorem{lem}[thm]{Lemma}
\newtheorem{cor}[thm]{Corollary}
\newtheorem{propo}[thm]{Proposition}
\newtheorem{clm}[thm]{Claim}
\newtheorem{defn}[thm]{Definition}
\newtheorem{rem}[thm]{Remark}
\newtheorem{egs}[thm]{Example}
\newtheorem{nte}[thm]{Note}

\newenvironment{theorem}{\begin{thm}}{\end{thm}}
\newenvironment{lemma}{\begin{lem}}{\end{lem}}
\newenvironment{corollary}{\begin{cor}}{\end{cor}}
\newenvironment{proposition}{\begin{propo}\begin{rm}}{\end{rm}\end{propo}}
\newenvironment{definition}{\begin{defn}}
{\end{defn}}
\newenvironment{claim}{\begin{clm}\begin{rm}}
{\end{rm}\end{clm}}
\newenvironment{remark}{\begin{rem}\begin{em}}
{\end{em}\end{rem}}
\newenvironment{example}{\begin{egs}\begin{em}}{\end{em}\end{egs}}
\newenvironment{note}{\begin{nte}\begin{rm}}{\end{rm}\end{nte}}

\newcommand{\secref}[1]{Section~\ref{#1}}
\newcommand{\apref}[1]{Appendix~\ref{#1}}
\newcommand{\thref}[1]{Theorem~\ref{#1}}
\newcommand{\defref}[1]{Definition~\ref{#1}}
\newcommand{\corref}[1]{Corollary~\ref{#1}}
\newcommand{\lemref}[1]{Lemma~\ref{#1}}
\newcommand{\clref}[1]{Claim~\ref{#1}}
\newcommand{\propref}[1]{Proposition~\ref{#1}}
\newcommand{\figref}[1]{Fig.~\ref{#1}}
\renewcommand{\eqref}[1]{\mbox{Equation~(\ref{#1})}}


\def\sqed{{\hspace{1pt}\rule[-1pt]{3pt}{9pt}}}
\def\qedsym{\hspace{1pt}\rule[-1pt]{3pt}{9pt}}
\newcommand{\mybox}{\raisebox{-1.5pt}{{\Large $\Box$}}}
\def\enddef{{\sqed}}


\newlength{\saveparindent}
\setlength{\saveparindent}{\parindent}
\newlength{\saveparskip}
\setlength{\saveparskip}{\parskip}




\def\qsym{\vrule width0.6ex height1em depth0ex}
\newcount\proofqeded
\newcount\proofended
\def\qed{{\hspace{1pt}\rule[-1pt]{3pt}{9pt}}
\end{rm}\addtolength{\parskip}{-0pt}
\setlength{\parindent}{\saveparindent}
\global\advance\proofqeded by 1 }
\def\qedenv{
\end{rm}\addtolength{\parskip}{-0pt}
\setlength{\parindent}{\saveparindent}
\global\advance\proofqeded by 1 }
\newenvironment{proof}%
 {\proofstart}%
 {\ifnum\proofqeded=\proofended~\qed\fi \global\advance\proofended by 1
  \medskip}
\newenvironment{proofenv}%
 {\proofenvstart}%
 {\ifnum\proofqeded=\proofended\qedenv\fi \global\advance\proofended by 1
  \medskip}
\makeatletter
\def\proofstart{\@ifnextchar[{\@oprf}{\@nprf}}
\def\proofenvstart{\@ifnextchar[{\@osprf}{\@nsprf}}
\def\@oprf[#1]{\begin{rm}\protect\vspace{6pt}\noindent{\bf Proof of #1:\ }%
\addtolength{\parskip}{5pt}\setlength{\parindent}{0pt}}
\def\@osprf[#1]{\begin{rm}\protect\vspace{6pt}\noindent
\addtolength{\parskip}{5pt}\setlength{\parindent}{0pt}}
\def\@nprf{\begin{rm}\protect\vspace{6pt}\noindent{\bf Proof:\ }%
\addtolength{\parskip}{5pt}\setlength{\parindent}{0pt}}
\def\@nsprf{\begin{rm}\protect\vspace{6pt}\noindent%
\addtolength{\parskip}{5pt}\setlength{\parindent}{0pt}}


\newcounter{ctr}
\newcounter{savectr}
\newcounter{ectr}
\newcounter{eectr}

\newenvironment{newenum}{%
\begin{list}{{\rm (\arabic{ctr})}\hfill}{\usecounter{ctr} \labelwidth=17pt%
\labelsep=3pt \leftmargin=20pt \topsep=2pt%
\setlength{\listparindent}{\saveparindent}%
\setlength{\parsep}{\saveparskip}%
\setlength{\itemsep}{1pt} }}{\end{list}}

\newenvironment{newenumbf}{%
\begin{list}{{\bf \arabic{ctr}.}\hfill}{\usecounter{ctr} \labelwidth=17pt%
\labelsep=3pt \leftmargin=20pt \topsep=1pt%
\setlength{\listparindent}{\saveparindent}%
\setlength{\parsep}{\saveparskip}%
\setlength{\itemsep}{1pt} }}{\end{list}}

\newenvironment{newitemize}{%
\begin{list}{$\bullet$\hfill}{\labelwidth=16pt%
\labelsep=5pt \leftmargin=21pt \topsep=3pt%
\setlength{\listparindent}{\saveparindent}%
\setlength{\parsep}{\saveparskip}%
\setlength{\itemsep}{3pt} }}{\end{list}}

\newlength{\savejot}
\setlength{\jot}{6pt}
\setlength{\savejot}{\jot}

\newenvironment{newmath}{\begin{displaymath}
\setlength{\abovedisplayskip}{6pt}
\setlength{\belowdisplayskip}{6pt}
\setlength{\abovedisplayshortskip}{8pt}
\setlength{\belowdisplayshortskip}{8pt}}{\end{displaymath}}

\newenvironment{neweqnarrays}{\begin{eqnarray*}%
\setlength{\abovedisplayskip}{-4pt}%
\setlength{\belowdisplayskip}{-4pt}%
\setlength{\abovedisplayshortskip}{-4pt}%
\setlength{\belowdisplayshortskip}{-4pt}%
\setlength{\jot}{-0.4in} }{\end{eqnarray*}}

\newenvironment{newequation}{\begin{equation}%
\setlength{\abovedisplayskip}{6pt}%
\setlength{\belowdisplayskip}{6pt}%
\setlength{\abovedisplayshortskip}{8pt}%
\setlength{\belowdisplayshortskip}{8pt} }{\end{equation}}

\newcommand{\heading}[1]{\vspace{5pt}\noindent\textsc{#1}}
\newcommand{\noskipheading}[1]{\noindent\textsc{#1}}

\newcommand{\tuple}[1]{\langle{#1}\rangle}
\newcommand{\bits}{\{0,1\}}
\newcommand{\cross}{\times}
\newcommand{\xor}{{\oplus}}
\newcommand{\Colon}{{:\:}}
\newcommand{\mystrut}{\rule{0em}{12pt}}
\newcommand{\namestrut}{\rule{0em}{20pt}}
\newcommand{\emptystring}{\varepsilon}

\newcommand{\calA}{{\cal A}}
\newcommand{\calB}{{\cal B}}
\newcommand{\calC}{{\cal C}}
\newcommand{\calF}{{\cal F}}
\newcommand{\calM}{{\cal M}}
\newcommand{\calO}{{\cal O}}
\newcommand{\calR}{{\cal R}}
\newcommand{\calH}{{\cal H}}
\newcommand{\calG}{{\cal G}}
\newcommand{\calD}{{\cal D}}
\newcommand{\calE}{{\cal E}}
\newcommand{\calP}{{\cal P}}
\newcommand{\calS}{{\cal S}}
\newcommand{\calU}{{\cal U}}
\newcommand{\calX}{{\cal X}}
\newcommand{\calY}{{\cal Y}}

\newcommand{\N}{{{\mathbb N}}	}
\newcommand{\Z}{{{\mathbb Z}}}
\newcommand{\R}{{{\mathbb R}}}
\newcommand{\goesto}{{\rightarrow}}
\newcommand{\then}{{\;;\;\;}}   % for [ ; ; ; : ] notation
\newcommand{\andthen}{{\;:\;\;}}

\newcommand{\suchthatt}{\: :\:}
\newcommand\nextt{\:;\:}
\newcommand{\sett}[1]{\{#1\}}
\newcommand{\set}[2]{\{\:#1 \suchthatt #2\:\}}
\newcommand{\setsize}[1]{\left|{#1}\right|}
\def\leqq{\;\leq\;}
\def\eqq{\;=\;}
\def\geqq{\;\geq\;}
\def\equivv{\;\equiv\;}
\def\prn#1{\left(#1\right)}
\newcommand{\getsr}{{\:{\leftarrow{\hspace*{-3pt}\raisebox{.75pt}{$\scriptscriptstyle\$$}}}\:}}

\newcommand{\Var}[1]{{\mbox{\bf Var}}[#1]}
\newcommand{\E}{\mathbf{E}}
\newcommand{\EE}[1]{{\E\left[{#1}\right]}}
\newcommand{\EEE}[2]{{\E_{#1}\left[{#2}\right]}}
\newcommand{\Prob}[1]{{\Pr\left[\,{#1}\,\right]}}
\newcommand{\condProb}[2]{{\Pr}\left[\,#1\,|\,#2\,\right]}
\newcommand{\concat}{{\,\|\,}}

\newcommand{\verylongleftarrow}[1]
      {\setlength{\unitlength}{.01in}
      \begin{picture}(#1,1) \put(#1,0){\vector(-1,0){#1}} \end{picture}}
\newcommand{\verylongrightarrow}[1]             %longleft and rightgoing arrows
      {\setlength{\unitlength}{.01in}           %for protocols
      \begin{picture}(#1,1) \put(0,0){\vector(1,0){#1}} \end{picture}}
\newcommand{\leftgoing}[2]{{\stackrel{{\displaystyle #2}} {\verylongleftarrow{#1}}}}
\newcommand{\rightgoing}[2]{{\stackrel{{\displaystyle #2}} {\verylongrightarrow{#1}}}}

\newcommand{\leftgoinga}[1]{\leftgoing{230}{#1} }
\newcommand{\rightgoinga}[1]{\rightgoing{230}{#1} }

\newcommand{\leftgoingb}[1]{\leftgoing{300}{#1} }
\newcommand{\rightgoingb}[1]{\rightgoing{300}{#1} }

\newcommand{\seq}{\!=\!}
\newcommand{\sminus}{\!-\!}
\newcommand{\splus}{\!+\!}

% ===================================================================
\newcommand{\kwfont}[1]{{\ensuremath{\mathrm{#1}}}}
\newcommand{\kwfunction}{{\kwfont{function}\ }}
\newcommand{\kwlabel}{{\kwfont{label}\ }}
\newcommand{\kwfor}{{\kwfont{For}\ }}
\newcommand{\kwand}{{\kwfont{and}\ }}
\newcommand{\kwor}{{\kwfont{or}\ }}
\newcommand{\kwnot}{{\kwfont{not}}}
\newcommand{\kwdo}{{\kwfont{do}\ }}
\newcommand{\kwreturn}{{\kwfont{\return}\ }}
\newcommand{\kwalgorithm}{{\ensuremath{\mathbf{Algorithm}\ }}}
\newcommand{\kwprotocol}{{\kwfont{Protocol}\ }}
\newcommand{\kwexperiment}{{\kwfont{Experiment}\ }}
\newcommand{\kwadversary}{{\kwfont{Adversary}\ }}
\newcommand{\kworacle}{{\kwfont{Oracle}\ }}
\newcommand{\kwuntil}{{\kwfont{until}\ }}
\newcommand{\kwrepeat}{{\kwfont{repeat}\ }}
\newcommand{\kwif}{{\kwfont{If}\ }}
\newcommand{\kwthen}{{\kwfont{then}\ }}
\newcommand{\kwelse}{{\kwfont{Else}\ }}
\newcommand{\kwabort}{{\kwfont{abort}\ }}
\newcommand{\kwgoto}{{\kwfont{goto}\ }}
\newcommand{\kwwhile}{{\kwfont{while}\ }}
\newcommand{\kwparse}{{\kwfont{parse}\ }}
\newcommand{\kwas}{{\kwfont{as}\ }}
\newcommand{\kwstatic}{{\kwfont{static}\ }}
\newcommand{\kwrun}{{\kwfont{run}\ }}
\newcommand{\kwbegin}{{\kwfont{begin}\ }}
\newcommand{\kwend}{{\kwfont{end}\ }}
\newcommand{\kwstart}{{\kwfont{start}\ }}
\newcommand{\kwcontinue}{{\kwfont{continue}\ }}
\newcommand{\kwdefine}{{\kwfont{define}\ }}
\newcommand{\kwflip}{{\kwfont{flip}\ }}
\newcommand{\kwlet}{{\kwfont{let}\ }}
\newcommand{\kwof}{{\kwfont{of}\ }}
\newcommand{\kwcase}{{\kwfont{case}\ }}
\newcommand{\kwswitch}{{\kwfont{switch}\ }}
\newcommand{\kwpick}{{\kwfont{pick}\ }}
\newcommand{\kwset}{{\kwfont{set}\ }}
\newcommand{\kwcompute}{{\kwfont{compute}\ }}
\newcommand{\comment}[1]{\hspace{5pt}{\small /$\!\!$/\ #1}}
\newcommand{\Comment}[1]{\hspace{5pt}{/$\!\!$/\ #1}}

\newcommand{\authnote}[2]{\ifnum\authnotes=1 \begin{center}\fbox{\begin{minipage}{\textwidth}
\textbf{#1 says:} #2\end{minipage}}\end{center} \fi}

\newcommand{\schemefont}[1]{\mathsf{#1}}
\newcommand{\schalg}[2]{#1.\mathsf{#2}}

\newcommand{\primfont}[1]{{\mathrm{#1}}}
\newcommand{\algfont}[1]{#1}
\newcommand{\advfont}[1]{#1}
\newcommand{\orfont}[1]{\mathsc{#1}}
\newcommand{\varfont}[1]{\mathit{#1}}
\newcommand{\randvarfont}[1]{\mathbf{#1}}
\newcommand{\constfont}[1]{\mathtt{#1}}
\newcommand{\notionfont}[1]{\mathcal{#1}}
\newcommand{\eventfont}[1]{{\mathsc{#1}}}
\newcommand{\vectorfont}[1]{\mathslbf{#1}}
\newcommand{\transformfont}[1]{\mathsf{#1}}
\newcommand{\gamefont}[1]{\mathrm{#1}}
\newcommand{\procfont}[1]{\mathsc{#1}}
\newcommand{\setfont}[1]{\mathsl{#1}}


\newcommand{\advA}{\calA}

\newcommand{\procIC}{\procfont{E}}
\newcommand{\procICinv}{\procfont{E}^{-1}}
\newcommand{\vecf}{\mathbf{f}}
\newcommand{\hashO}{\procfont{Hash}}
\newcommand{\hashSimO}{\procfont{HashSim}}
\newcommand{\progO}{\procfont{Prog}}
\newcommand{\hashOO}{\procfont{HashO}}
\newcommand{\hashIO}{\procfont{HashI}}
\newcommand{\challengeO}{\procfont{Challenge}}
\newcommand{\refreshO}{\procfont{Refresh}}
\newcommand{\secIn}{1^\secParam}
\newcommand{\secParam}{\lambda}
\newcommand{\vech}{\mathbf{h}}
\newcommand{\setf}{\setfont{F}}
\newcommand{\domain}{\setfont{D}}
\newcommand{\range}{\setfont{R}}

\newcommand{\true}{\mathsf{true}}
\newcommand{\false}{\mathsf{false}}

\newcommand{\Bad}{\mathsf{Bad}}

\newcommand{\gameName}[1]{\underline{Game #1}\\[2pt]}

\newcommand{\gamesfontsize}{\small}
\newcommand{\ind}{\hspace*{10pt}}

\newcommand{\mpage}[2]{\begin{minipage}{#1\textwidth}
    #2 \end{minipage}}

\newcommand{\oneCol}[2]{
\begin{center}\framebox{\begin{tabular}{c}
\begin{minipage}[t]{#1\textwidth}\setstretch{1.2}\gamesfontsize
#2
\end{minipage} 
\end{tabular}}\end{center}}




\newcommand{\twoColsNoDivide}[4]{
\begin{center}
        \framebox{
        \begin{tabular}{c@{\hspace*{.4em}}c@{\hspace*{.4em}}c}
        \begin{minipage}[t]{#1\textwidth}\setstretch{1.1}\gamesfontsize #3 \end{minipage}
        &
        \begin{minipage}[t]{#2\textwidth}\setstretch{1.1}\gamesfontsize #4 \end{minipage}
        \end{tabular}
        }
\end{center}
}


\newcommand{\twoCols}[4]{
\begin{center}
        \framebox{
        \begin{tabular}{c@{\hspace*{.4em}}|c@{\hspace*{.4em}}c}
        \begin{minipage}[t]{#1\textwidth}\setstretch{1.1}\gamesfontsize #3 \end{minipage}
        &
        \begin{minipage}[t]{#2\textwidth}\setstretch{1.1}\gamesfontsize #4 \end{minipage}
        \end{tabular}
        }
\end{center}
}

\newcommand{\twoColsTwoRows}[6]{
\begin{center}
        \framebox{
        \begin{tabular}{c@{\hspace*{.4em}}|c@{\hspace*{.4em}}c}
        \begin{minipage}[t]{#1\textwidth}\setstretch{1.1}\gamesfontsize #3 \end{minipage}
        &
        \begin{minipage}[t]{#2\textwidth}\setstretch{1.1}\gamesfontsize #4 \end{minipage}
        \\ \hline
        \begin{minipage}[t]{#1\textwidth}\setstretch{1.1}\gamesfontsize #5
        \end{minipage} &
        \begin{minipage}[t]{#2\textwidth}\setstretch{1.1}\gamesfontsize #6        
        \end{minipage}
        \end{tabular}
        }
\end{center}
}
\newcommand{\threeCols}[6]{
\begin{center}
        \framebox{
        \begin{tabular}{@{\hspace{-0.2em}}c@{\hspace{0.2em}}|@{\hspace{0.2em}}c@{\hspace{0.2em}}|@{\hspace{0.2em}}c@{\hspace{0.2em}}}
        \begin{minipage}[t]{#1\textwidth}\setstretch{1.1}\gamesfontsize #4
        \end{minipage} &
        \begin{minipage}[t]{#2\textwidth}\setstretch{1.1}\gamesfontsize #5
        \end{minipage} &
        \begin{minipage}[t]{#3\textwidth}\setstretch{1.1}\gamesfontsize #6
        \end{minipage}
        \end{tabular}
        }
\end{center}
}

\newcommand{\threeColstwoSplit}[6]{
\begin{center}
        \framebox{
        \begin{tabular}{@{\hspace{-0.2em}}c@{\hspace{0.2em}}|@{\hspace{0.2em}}c@{\hspace{0.2em}}@{\hspace{0.2em}}c@{\hspace{0.2em}}}
        \begin{minipage}[t]{#1\textwidth}\setstretch{1.1}\gamesfontsize #4
        \end{minipage} &
        \begin{minipage}[t]{#2\textwidth}\setstretch{1.1}\gamesfontsize #5
        \end{minipage} &
        \begin{minipage}[t]{#3\textwidth}\setstretch{1.1}\gamesfontsize #6
        \end{minipage}
        \end{tabular}
        }
\end{center}
}

\newcommand{\fourColstwoSplit}[8]{
\begin{center}
        \framebox{
        \begin{tabular}{@{\hspace{-0.2em}}c@{\hspace{0.2em}}|@{\hspace{0.2em}}c@{\hspace{0.2em}}@{\hspace{0.2em}}c@{\hspace{0.2em}}@{\hspace{0.2em}}c@{\hspace{0.2em}}}
        \begin{minipage}[t]{#1\textwidth}\setstretch{1.1}\gamesfontsize #5
        \end{minipage} &
        \begin{minipage}[t]{#2\textwidth}\setstretch{1.1}\gamesfontsize #6
        \end{minipage} &
        \begin{minipage}[t]{#3\textwidth}\setstretch{1.1}\gamesfontsize #7
        \end{minipage}  &
        \begin{minipage}[t]{#4\textwidth}\setstretch{1.1}\gamesfontsize #8
        \end{minipage}
        \end{tabular}
        }
\end{center}
}


\newcommand{\threeColsSplit}[6]{
\begin{center}
        \framebox{
        \begin{tabular}{@{\hspace{-0.2em}}c@{\hspace{0.2em}}||@{\hspace{0.2em}}c@{\hspace{0.2em}}|@{\hspace{0.2em}}c@{\hspace{0.2em}}}
        \begin{minipage}[t]{#1\textwidth}\setstretch{1.1}\gamesfontsize #4
        \end{minipage} &
        \begin{minipage}[t]{#2\textwidth}\setstretch{1.1}\gamesfontsize #5
        \end{minipage} &
        \begin{minipage}[t]{#3\textwidth}\setstretch{1.1}\gamesfontsize #6
        \end{minipage}
        \end{tabular}
        }
\end{center}
}


\newcommand{\fourCols}[8]{
\begin{center}
        \framebox{
        \begin{tabular}{@{\hspace{-0.2em}}c@{\hspace{0.2em}}|@{\hspace{0.2em}}c@{\hspace{0.2em}}|@{\hspace{0.2em}}c@{\hspace{0.2em}}|@{\hspace{0.2em}}c@{\hspace{0.2em}}}
        \begin{minipage}[t]{#1\textwidth}\setstretch{1.2}\gamesfontsize #5
        \end{minipage} &
        \begin{minipage}[t]{#2\textwidth}\setstretch{1.2}\gamesfontsize #6
        \end{minipage} &
        \begin{minipage}[t]{#3\textwidth}\setstretch{1.2}\gamesfontsize #7
        \end{minipage}&
        \begin{minipage}[t]{#4\textwidth}\setstretch{1.2}\gamesfontsize #8
        \end{minipage}
        \end{tabular}
        }
\end{center}
}


\newcommand{\threeColsTwoRows}[9]{
\begin{center}
        \framebox{
        \begin{tabular}{c@{\hspace*{.4em}}|c@{\hspace*{.4em}}|c@{\hspace*{.4em}}}
        \begin{minipage}[t]{#1\textwidth}\setstretch{1.2}\gamesfontsize #4
        \end{minipage} &
        \begin{minipage}[t]{#2\textwidth}\setstretch{1.2}\gamesfontsize #5
        \end{minipage} &
        \begin{minipage}[t]{#3\textwidth}\setstretch{1.2}\gamesfontsize #6
        \end{minipage}\\\hline 
        \begin{minipage}[t]{#1\textwidth}\setstretch{1.2}\gamesfontsize
         #7 \end{minipage} &
        \begin{minipage}[t]{#2\textwidth}\setstretch{1.2}\gamesfontsize
         #8 \end{minipage} &
        \begin{minipage}[t]{#3\textwidth}\setstretch{1.2}\gamesfontsize
         #9
        \end{minipage}
        \end{tabular}
        }
\end{center}
}

\newcommand{\threeColsNoBox}[3]{
\begin{center}\begin{tabular}{c|c|c}
\begin{minipage}[t]{1in}\begin{tabbing}
12\=12\=12\=\kill
#1
\end{tabbing}\end{minipage} &
\begin{minipage}[t]{1in}\begin{tabbing}
12\=12\=12\=\kill
#2
\end{tabbing}\end{minipage} &
\begin{minipage}[t]{1in}\begin{tabbing}
12\=12\=12\=\kill
#3
\end{tabbing}\end{minipage} 
\end{tabular}\end{center}}

\newcommand{\fourColsNoBox}[4]{
\begin{center}\begin{tabular}{c|c|c|c}
\begin{minipage}[t]{1in}\begin{tabbing}
12\=12\=12\=\kill
#1
\end{tabbing}\end{minipage} &
\begin{minipage}[t]{1in}\begin{tabbing}
12\=12\=12\=\kill
#2
\end{tabbing}\end{minipage} &
\begin{minipage}[t]{1in}\begin{tabbing}
12\=12\=12\=\kill
#3
\end{tabbing}\end{minipage} &
\begin{minipage}[t]{1in}\begin{tabbing}
12\=12\=12\=\kill
#4
\end{tabbing}\end{minipage} 
\end{tabular}\end{center}}


\newcommand{\twoColsNoBox}[2]{
\begin{center}\begin{tabular}{c|c}
\begin{minipage}[t]{1in}\begin{tabbing}
12\=12\=12\=\kill
#1
\end{tabbing}\end{minipage} &
\begin{minipage}[t]{1in}\begin{tabbing}
12\=12\=12\=\kill
#2
\end{tabbing}\end{minipage} 
\end{tabular}\end{center}}


\newcommand{\twoColsNoBoxNoDivide}[2]{
\begin{center}\begin{tabular}{ccc}
\begin{minipage}[t]{1in}\begin{tabbing}
12\=12\=12\=\kill
#1
\end{tabbing}\end{minipage} & \hspace{10pt} &
\begin{minipage}[t]{1in}\begin{tabbing}
12\=12\=12\=\kill
#2
\end{tabbing}\end{minipage} 
\end{tabular}\end{center}}


\newcommand{\threeColsNoBoxOneDivide}[3]{
\begin{center}\begin{tabular}{ccc|c}
\begin{minipage}[t]{1in}\begin{tabbing}
12\=12\=12\=\kill
#1
\end{tabbing}\end{minipage} & \hspace{10pt} &
\begin{minipage}[t]{1in}\begin{tabbing}
12\=12\=12\=\kill
#2
\end{tabbing}\end{minipage} & 
\begin{minipage}[t]{1in}\begin{tabbing}
12\=12\=12\=\kill
#2
\end{tabbing}\end{minipage}
\end{tabular}\end{center}}


\newcommand{\oneColNoBox}[1]{
\begin{center}\begin{tabular}{c}
\begin{minipage}[t]{1in}\begin{tabbing}
12\=12\=12\=\kill
#1
\end{tabbing}\end{minipage} 
\end{tabular}\end{center}}

\newcommand{\aeScheme}{\schemefont{AE}}
\newcommand{\eScheme}{\mathsf{ES}}
\newcommand{\EncO}{\procfont{Enc}}
\newcommand{\VfO}{\procfont{Vf}}
\newcommand{\NewO}{\procfont{New}}
\newcommand{\LRO}{\procfont{LR}}
\newcommand{\LROSim}{\procfont{LRSim}}
\newcommand{\RoRO}{\procfont{RoR}}
\newcommand{\EncRealO}{\procfont{Enc}}
\newcommand{\Fn}{\procfont{Fn}}

\newcommand{\genAdv}[3]{\mathbf{Adv}^{\rm #1}_{#2}(#3)}

\newcommand{\HF}{\mathsf{H}}
\newcommand{\FF}{\mathsf{F}}
\newcommand{\DOM}[1]{#1.\mathsf{Dom}}
\newcommand{\RNG}[1]{#1.\mathsf{Rng}}
\newcommand{\KEYS}[1]{#1.\mathsf{Keys}}
\newcommand{\KLength}[1]{#1.\mathsf{kl}}
\newcommand{\OLength}[1]{#1.\mathsf{ol}}
\newcommand{\ILength}[1]{#1.\mathsf{il}}
\newcommand{\GF}{\mathsf{G}}
\newcommand{\oGF}{\overline{\mathsf{G}}}
\newcommand{\HKey}{\mathsl{hk}}


\newcommand{\indAdv}[2]{\genAdv{ind}{#1}{#2}}
\newcommand{\prfAdv}[2]{\genAdv{prf}{#1}{#2}}
\newcommand{\rorAdv}[2]{\genAdv{ind\$}{#1}{#2}}
\newcommand{\aeAdv}[2]{\genAdv{mu\mbox{-}ind}{#1}{#2}}
\newcommand{\krAdv}[2]{\genAdv{mu\mbox{-}kr}{#1}{#2}}
\newcommand{\intAdv}[2]{\genAdv{int}{#1}{#2}}

\newcommand{\gameG}{\gamefont{G}}
\newcommand{\game}[2]{\mathbf{G}^{\rm #1}_{#2}}
\newcommand{\gAE}[1]{\game{mu\mbox{-}ind}{#1}}
\newcommand{\gKR}{\game{mu\mbox{-}kr}}
\newcommand{\gAEKDM}{\gamefont{AE\mbox{-}KDM}}
\newcommand{\gKDMCCA}{\gamefont{KDM\mbox{-}CCA}}
\newcommand{\KDMCCA}{\mathrm{KDM\mbox{-}CCA}}
\newcommand{\gFKDMReal}{\gamefont{KDM}}
\newcommand{\gFKDM}{\gamefont{KDM}}
\newcommand{\gFKDMRand}{\gamefont{KDM^*}}
\newcommand{\gKDMP}{\gamefont{KDMP}}
\newcommand{\gKDMF}{\gamefont{KDM}^{-}}
\newcommand{\gKDMFP}{\gamefont{KDMP}^{-}}

\newcommand{\pkeScheme}{\schemefont{PKE}}
\newcommand{\PKEkg}{\schalg{PKE}{Kg}}
\newcommand{\PKEenc}{\schalg{PKE}{Enc}}
\newcommand{\PKEdec}{\schalg{PKE}{Dec}}
\newcommand{\PKEIL}{\schalg{PKE}{IL}}
\newcommand{\PKEil}{\schalg{PKE}{il}}
\newcommand{\PKEOL}{\schalg{PKE}{OL}}
\newcommand{\PKEol}{\schalg{PKE}{ol}}
\newcommand{\PKErl}{\schalg{PKE}{rl}}

\newcommand{\REScheme}{\schemefont{RE}}
\newcommand{\REkg}{\schalg{RE}{Kg}}
\newcommand{\REenc}{\schalg{RE}{Enc}}
\newcommand{\REdec}{\schalg{RE}{Dec}}
\newcommand{\REIL}{\schalg{RE}{IL}}
\newcommand{\REil}{\schalg{RE}{il}}
\newcommand{\REOL}{\schalg{RE}{OL}}
\newcommand{\REol}{\schalg{RE}{ol}}
\newcommand{\RErl}{\schalg{RE}{rl}}

\newcommand{\deScheme}{\schemefont{DE}}
\newcommand{\DEIL}{\schalg{DE}{IL}}
\newcommand{\DEil}{\schalg{DE}{il}}
\newcommand{\DEOL}{\schalg{DE}{OL}}
\newcommand{\DEol}{\schalg{DE}{ol}}
\newcommand{\DEkg}{\schalg{DE}{Kg}}
\newcommand{\DEenc}{\schalg{DE}{Enc}}
\newcommand{\DEdec}{\schalg{DE}{Dec}}

\newcommand{\seKg}{\schalg{\seScheme}{Kg}}
\newcommand{\seEnc}{\schalg{\seScheme}{Enc}}
\newcommand{\seDec}{\schalg{\seScheme}{Dec}}
\newcommand{\SEcl}{\schalg{\seScheme}{cl}}
\newcommand{\seSEnc}{\schalg{\seScheme}{SEnc}}
\newcommand{\seSP}{\schalg{\seScheme}{SP}}
\newcommand{\seScheme}{\schemefont{AE}}
\newcommand{\sekl}{\schalg{\seScheme}{kl}}
\newcommand{\sevl}{\schalg{\seScheme}{vl}}
\newcommand{\serl}{\schalg{\seScheme}{rl}}
\newcommand{\seNS}{\schalg{\seScheme}{NS}}
\newcommand{\msg}{M}
\newcommand{\ciph}{C}
\newcommand{\header}{H}
\newcommand{\nonce}{N}
\newcommand{\iv}{N}
\newcommand{\nkey}{\mathsl{K}}
\newcommand{\seckl}{\seScheme.\mathsf{ckl}}
\newcommand{\secbl}{\seScheme.\mathsf{bl}}


\newcommand{\setX}{X}








